\begin{titlepage}
\begin{center}
\vspace*{1cm}
\Huge

\vspace{0.5cm}
\LARGE
Does Women Care More?\\
The relationship between women's participation in law-making and national health expenditure
\\

\vspace{1.5 cm}
\textbf{Zhiyuan Jiang\\}
\vfill

\vspace{0.8cm}
 
\Large
\today
\end{center}
\end{titlepage}


\section*{Abstract}

This paper is trying to investigate the relationship between a country's budget size in the healthcare sector and the ratio of women in the country's legislature. 
Using panel data from 122 countries across 20 years and a fixed-effect model, we found in middle-lower income countries and countries with a deficient democracy system, an increasing proportion of female lawmakers in the parliament is positively correlated with the country's healthcare expenditure.
In other types of countries, this relationship is not significant.
This paper also discussed such relationships in different scenarios, such as in countries with different healthcare systems or considering the gender of the country's leader, and the relationship is insignificant. 
The rationale behind such results, we believe, is because only in certain types of countries, here refer to countries with middle-lower income levels or a deficient democratic system, female politicians could have space to exert their influence. 
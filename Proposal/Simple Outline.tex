\documentclass[12pt]{article}
\usepackage{amsmath}
\usepackage{mathptmx}
\usepackage{setspace}
\usepackage{amssymb}
\usepackage{float}
\usepackage{graphicx}
\usepackage{booktabs}
\usepackage{subfigure}
\usepackage[a4paper, margin=2.5cm]{geometry}
\usepackage[title]{appendix}
\usepackage{bm}
\usepackage{apacite}
\usepackage{diagbox}
\usepackage{longtable}
\usepackage{pdflscape}
\usepackage{rotating}
\usepackage{fancyhdr}

% Assign Variable
\newcommand{\numberofweek}{}
\newcommand{\titleofweek}{}


\linespread{1}

\title{ECON642\\ Term Paper Idea \numberofweek  \titleofweek}
\vspace{-3\baselineskip}
\author{Zhiyuan Jiang (Unique ID: 1082117)}
%\date{\today}

\begin{document}
\maketitle

\section*{Main Idea:}
Discussing the relationship between economic growth and women's participation in the decision-making, especially the legislative branch. 

\section*{Research Object:}
The research will be divided into two parts.\\

The first part will focus on developed countries.
First will compare the economic growth and other indexes relate to the quality of life to the ratio of women's participation in the law-making process.
The comparison groups including:
\begin{enumerate}
	\item High-income countries with a higher than average ratio of female lawmakers: Scandinavia countries (Finland, Norway, etc), Belgium, New Zealand, Taiwan.
	\item High-income countries with a low or around the average ratio of female lawmakers: South Korea, Japan, U.S., Ireland, Greece
\end{enumerate}

For the second part, I will use Rwanda, the country with the highest ratio of female lawmakers in the world, as the benchmark, to compare with other low or medium-income countries to see the effect of women's participation rate in parliament, with regard to economic growth and quality of life. 
Other than Rwanda, I will collect data for four different types of countries.


\begin{enumerate}
	\item High economic growth and high women participation rate
	\item High economic growth and low women participation rate
	\item Low economic growth and high women participation rate
	\item Low economic growth and low women participation rate
\end{enumerate}
A detailed list of countries is to be determined.


\section*{Source of Data}
World bank has data about the proportion of women participating in the national parliament.
World Bank also has a series of other data to measure development level and economic growth.


Data will be restricted within a certain time frame, especially considering the turbulence caused by the pandemic, the data is likely to be a time series (for instance 1979 - 2019).

\section*{Methodology}


Simple linear regression will be used in this study.
For comparison, I will add different control variables to see the influence.
Potential control variables include:
\begin{itemize}
	\item Democracy level
	\item Economic foundation (measured as the GDP level at the beginning year of the data)
	\item Women's education level (like the enrollment rate for school girls)
	\item general labor participation rate of women
	\item etc.
\end{itemize}

\section*{Other Ideas/Comments}
\begin{enumerate}
	\item Economic growth will be a too general topic to be discussed in this paper, and the GDP growth is obviously influenced by too many factors. Therefore, discussing the relationship between the ratio of female lawmakers and other smaller variables may be better. For instance, the relationship between the ratio of female lawmakers and the national budget of education, or the deployment of health care policy like the covid-19 vaccination rate.
	\item Some other econometrics methods shall be carefully considered for this topic. For instance IV and 2SLS when the endogenous problem presents, or if the data is available, using the panel data model. 
\end{enumerate}

\end{document}